\def\thelstlisting{}

%不需要区分奇偶页的请使用下面一行
\documentclass[a4paper,AutoFakeBold,oneside,12pt]{book}
%需要区分奇偶页的(即每一章第一页一定在奇数页上)请使用下面一行
%\documentclass[a4paper,AutoFakeBold,openright,12pt]{book}
\usepackage{BUPTthesisbachelor}
\usepackage{setspace}

%\lstdefinestyle{sharpc}{language=[Sharp]C, frame=lrtb, rulecolor=\color{blue!80!black}}


%%%%%%%%%%%%%%%%%%%%%%%%% Begin Documents %%%%%%%%%%%%%%%%%%%%%%%%%%
\begin{document}

% 封面
\blankmatter
\includepdf[pages=-]{docs/cover.pdf}  

% 任务书
\blankmatter
\includepdf[pages=-]{docs/task.pdf}  

% 成绩评定表
\blankmatter
\includepdf[pages=-]{docs/scoreTable.pdf}  

% 诚信声明
\blankmatter
\includepdf[pages=-]{docs/statement.pdf} 

%不需要区分奇偶页的请使用下面一行
\documentclass[a4paper,AutoFakeBold,oneside,12pt]{book}
%需要区分奇偶页的(即每一章第一页一定在奇数页上)请使用下面一行
%\documentclass[a4paper,AutoFakeBold,openright,12pt]{book}
\usepackage{BUPTthesisbachelor}
\usepackage{setspace}

%%%%%%%%%%%%%%%%%%%%%%%%% Begin Documents %%%%%%%%%%%%%%%%%%%%%%%%%%
\begin{document}

% 加入封面诚信保证书等

% 封面
\blankmatter
\includepdf[pages=-]{docs/cover.pdf}  

% 任务书
\blankmatter
\includepdf[pages=-]{docs/task.pdf}  

% 成绩评定表
\blankmatter
\includepdf[pages=-]{docs/scoreTable.pdf}  

% 诚信声明
\blankmatter
\includepdf[pages=-]{docs/statement.pdf} 


%不需要区分奇偶页的请使用下面一行
\documentclass[a4paper,AutoFakeBold,oneside,12pt]{book}
%需要区分奇偶页的(即每一章第一页一定在奇数页上)请使用下面一行
%\documentclass[a4paper,AutoFakeBold,openright,12pt]{book}
\usepackage{BUPTthesisbachelor}
\usepackage{setspace}

%%%%%%%%%%%%%%%%%%%%%%%%% Begin Documents %%%%%%%%%%%%%%%%%%%%%%%%%%
\begin{document}

% 加入封面诚信保证书等

% 封面
\blankmatter
\includepdf[pages=-]{docs/cover.pdf}  

% 任务书
\blankmatter
\includepdf[pages=-]{docs/task.pdf}  

% 成绩评定表
\blankmatter
\includepdf[pages=-]{docs/scoreTable.pdf}  

% 诚信声明
\blankmatter
\includepdf[pages=-]{docs/statement.pdf} 


%不需要区分奇偶页的请使用下面一行
\documentclass[a4paper,AutoFakeBold,oneside,12pt]{book}
%需要区分奇偶页的(即每一章第一页一定在奇数页上)请使用下面一行
%\documentclass[a4paper,AutoFakeBold,openright,12pt]{book}
\usepackage{BUPTthesisbachelor}
\usepackage{setspace}

%%%%%%%%%%%%%%%%%%%%%%%%% Begin Documents %%%%%%%%%%%%%%%%%%%%%%%%%%
\begin{document}

% 加入封面诚信保证书等
\input{contents/foreword.tex}

\input{main.cfg}    % Main items 
\include{abstract}  % Abstract

% 加入目录
\fancypagestyle{plain}{\pagestyle{frontmatter}}
\frontmatter
\tableofcontents % Content

% 正文
\newpage\mainmatter
\fancypagestyle{plain}{\pagestyle{mainmatter}}

%%%%%%%%%%%%%%%%%%%%%%%%%%%%% Main Area %%%%%%%%%%%%%%%%%%%%%%%%%%%%

% 加入正文建议按章节组织,可将示例文件备份,有不会的用法再来参考
\input{contents/chapter1.tex}

\input{contents/chapter2.tex}

%%%%%%%%%%%%%%%%%%%%%%% Main Area ENDs Here %%%%%%%%%%%%%%%%%%%%%%%%

\begin{nopagenumber}
% Reference
\clearpage\phantomsection\addcontentsline{toc}{chapter}{参考文献}
\bibliographystyle{buptbachelor}
\refbodyfont{\bibliography{ref}}

% Thanks to page
\clearpage
\chapter{致\qquad{}谢}
\normalsize\thankwords

% 加入附录
\input{contents/appendix.tex}

% 加入外文及翻译
\input{contents/translate.tex}

% 加入开题报告中期检查表和教师指导毕业设计表可按需求注释
\input{contents/epilogue.tex}

\end{nopagenumber}

\end{document}
    % Main items 
\include{abstract}  % Abstract

% 加入目录
\fancypagestyle{plain}{\pagestyle{frontmatter}}
\frontmatter
\tableofcontents % Content

% 正文
\newpage\mainmatter
\fancypagestyle{plain}{\pagestyle{mainmatter}}

%%%%%%%%%%%%%%%%%%%%%%%%%%%%% Main Area %%%%%%%%%%%%%%%%%%%%%%%%%%%%

% 加入正文建议按章节组织,可将示例文件备份,有不会的用法再来参考
\input{contents/chapter1.tex}

\chapter{为了目录撑到第二页}
\section{我不得不再添加一点内容}
\section{尽管这些章节一点正文都没有}
\section{是的}
\section{真的没有}
\section{我已经不知道说什么了}
\section{如果有,我们就祝愿一下学校教务处什么时候转变一下思维}
\section{把控制格式这种事情往前做}
\section{不要总是觉得折磨学生是合理的}
\section{你拿着教学管理岗位的工资}
\section{你需要折磨一下你自己才对}
\section{不要觉得我对别人要求太高,对自己太低}
\section{我对自己要求低的话也不至于想要修订这份模板}


%%%%%%%%%%%%%%%%%%%%%%% Main Area ENDs Here %%%%%%%%%%%%%%%%%%%%%%%%

\begin{nopagenumber}
% Reference
\clearpage\phantomsection\addcontentsline{toc}{chapter}{参考文献}
\bibliographystyle{buptbachelor}
\refbodyfont{\bibliography{ref}}

% Thanks to page
\clearpage
\chapter{致\qquad{}谢}
\normalsize\thankwords

% 加入附录
\input{contents/appendix.tex}

% 加入外文及翻译
\input{contents/translate.tex}

% 加入开题报告中期检查表和教师指导毕业设计表可按需求注释
% 开题报告
\blankmatter
\phantomsection\addcontentsline{toc}{chapter}{开\quad{}题\quad{}报\quad{}告}
\includepdf[pages=-]{docs/openingReport.pdf}

% 中期检查表
\blankmatter
\phantomsection\addcontentsline{toc}{chapter}{中\quad{}期\quad{}检\quad{}查\quad{}表}
\includepdf[pages=-]{docs/interimReport.pdf}

% 教师指导毕业设计(论文)记录表
\blankmatter
\phantomsection\addcontentsline{toc}{chapter}{教师指导毕业设计(论文)记录表}
\includepdf[pages=-]{docs/guidance.pdf}


\end{nopagenumber}

\end{document}
    % Main items 
\include{abstract}  % Abstract

% 加入目录
\fancypagestyle{plain}{\pagestyle{frontmatter}}
\frontmatter
\tableofcontents % Content

% 正文
\newpage\mainmatter
\fancypagestyle{plain}{\pagestyle{mainmatter}}

%%%%%%%%%%%%%%%%%%%%%%%%%%%%% Main Area %%%%%%%%%%%%%%%%%%%%%%%%%%%%

% 加入正文建议按章节组织,可将示例文件备份,有不会的用法再来参考
\input{contents/chapter1.tex}

\chapter{为了目录撑到第二页}
\section{我不得不再添加一点内容}
\section{尽管这些章节一点正文都没有}
\section{是的}
\section{真的没有}
\section{我已经不知道说什么了}
\section{如果有,我们就祝愿一下学校教务处什么时候转变一下思维}
\section{把控制格式这种事情往前做}
\section{不要总是觉得折磨学生是合理的}
\section{你拿着教学管理岗位的工资}
\section{你需要折磨一下你自己才对}
\section{不要觉得我对别人要求太高,对自己太低}
\section{我对自己要求低的话也不至于想要修订这份模板}


%%%%%%%%%%%%%%%%%%%%%%% Main Area ENDs Here %%%%%%%%%%%%%%%%%%%%%%%%

\begin{nopagenumber}
% Reference
\clearpage\phantomsection\addcontentsline{toc}{chapter}{参考文献}
\bibliographystyle{buptbachelor}
\refbodyfont{\bibliography{ref}}

% Thanks to page
\clearpage
\chapter{致\qquad{}谢}
\normalsize\thankwords

% 加入附录
\input{contents/appendix.tex}

% 加入外文及翻译
\input{contents/translate.tex}

% 加入开题报告中期检查表和教师指导毕业设计表可按需求注释
% 开题报告
\blankmatter
\phantomsection\addcontentsline{toc}{chapter}{开\quad{}题\quad{}报\quad{}告}
\includepdf[pages=-]{docs/openingReport.pdf}

% 中期检查表
\blankmatter
\phantomsection\addcontentsline{toc}{chapter}{中\quad{}期\quad{}检\quad{}查\quad{}表}
\includepdf[pages=-]{docs/interimReport.pdf}

% 教师指导毕业设计(论文)记录表
\blankmatter
\phantomsection\addcontentsline{toc}{chapter}{教师指导毕业设计(论文)记录表}
\includepdf[pages=-]{docs/guidance.pdf}


\end{nopagenumber}

\end{document}
    % Main items 
\include{abstract}  % Abstract

\fancypagestyle{plain}{\pagestyle{frontmatter}}
\frontmatter
\tableofcontents % Content

% 正文
\newpage\mainmatter
\fancypagestyle{plain}{\pagestyle{mainmatter}}

%\let\cleardoublepagebak=\cleardoublepage
%\let\cleardoublepage\relax % Make new chapter stay on old page

%%%%%%%%%%%%%%%%%%%%%%%%%%%%% Main Area %%%%%%%%%%%%%%%%%%%%%%%%%%%%

\chapter{绪论}
\section{项目背景}
随着图像生成技术的不断发展,图像编辑作为其中的关键技术之一,应用广泛,涵盖了媒体娱乐、数字营销和智能医疗等多个领域。然而,传统的图像编辑模型存在着交互性差和生成图像质量受限的问题,迫使我们探索更先进的方法以提高图像生成的质量和用户交互性。通过深度学习和语言大模型的结合,我们有望构建一个创新的交互式图像编辑系统,为图像编辑领域带来新的可能性。
\section{项目意义}
通过结合深度学习和语言大模型,可以为图像编辑领域带来创新和进步。这样的研究对于推动图像处理领域的发展具有重要意义,可以应用于媒体娱乐、数字营销、智能医疗等多个领域,为用户提供更优质的图像编辑体验,促进相关产业的发展。
\section{项目内容}
\buptfigure[width=0.7\textwidth]{pictures/arch.png}{}{arch_TMP}

该项目主要实现了GUI、middleware,并对Stable Diffusion和ChatGLM2-6B进行修改与适配。
各个模块之间的关系如 图1-1 所示。

通过在基于Stable Diffusion的开源项目stable-diffusion-webui上进行扩展,
本项目实现了通过API调用多种Stable Diffusion模型对图像进行修改的功能。

通过对ChatGLM2-6B进行微调,本项目实现了通过API调用针对本任务微调过的ChatGLM2-6B模型。

通过调用OpenAI的API,本项目实现了多个功能:通过GPT4V生成图像修改的建议、
通过GPT3.5Turbo辅助生成微调大语言模型的数据集和图像修改的指令、
通过DALL-E2实现在Stable Diffusion不可用时作为替代模型对图像进行修改。

GUI主要通过python语言实现,其构建了一个直观、易于使用的用户交互界面。
在消耗大量计算资源的任务上,GUI会通过API对middleware发出请求,
减少了用户侧对计算资源的依赖,降低了用户的使用门槛。

middleware使用golang语言搭建了一个后端服务,其接入了Stable Diffusion、ChatGLM2-6B、OpenAI的API,并将这些API进行整合后向GUI提供API。middleware的建立实现了一对多服务的能力,提高了GUI调用多方API的便利性,同时通过统一配置提高了系统的可维护性。



\chapter{总体方案设计}
\section{GUI}
GUI虽然承担计算任务较少,但却是承载本项目结构与逻辑的关键部分。通过使用符合规则的指令作为中枢,
GUI打通了大语言模型和图像生成模型之间的壁垒,使基于LLM的创新交互式图像编辑系统成为可能。GUI的模块构成如 表2-1 :
\begin{bupttable}{GUI模块}{table_gui_modules}
    \begin{tabular}{|l|l|}
        \hline \textbf{模块} & \textbf{描述} \\
        \hline BaseImage & 接受上传的原始图片并预览  \\
        \hline EditedImage & 预览修改后的图片 \\
        \hline Operation Board & 执行指令  \\
		\hline Settings & 对系统进行设置  \\
		\hline Chat & 与大语言模型交互的聊天界面  \\
		\hline Edit Image & 对图像进行自定义遮罩和换脸等操作  \\
		\hline Auto & 执行自动化操作  \\
		\hline Manual & 系统使用说明  \\
        \hline
    \end{tabular}
\end{bupttable}

用户首先上传需要修改的图片,然后可在Chat模块中选择不同的大语言模型进行交互并得到相应的指令,
最后在Operation Board模块中选择指令执行或一键全部执行。如果对自动生成的遮罩不满意,
可在Edit Image中对遮罩进行修改。

在Auto模块中,用户可通过选择多张图片批量生成满足微调大语言模型微调所需的数据。
其会循环地从给定的图片集中随机选择图片继续分割,将分割后的结果和特定的prompt通过GPT3.5Turbo生成对应的修改建议,
再将分割的结果、生成的建议通过GPT3.5Turbo生成指令。


\section{middleware}
middleware是项目的核心组件,通过整合多个平台的API,为GUI提供统一的、简单易接入的API服务。其主要特点包括但不限于:
1. API整合:middleware整合了多个平台的API,包括图像生成模型、语言模型等,使得GUI可以通过统一的接口调用不同功能模块;
2. 统一风格:middleware设计了统一的API风格和路由规范,使得GUI可以轻松使用API服务,提高开发效率;
3. 简单易接入:middleware提供了简单易用的API服务,GUI无需关注具体实现细节,只需按照简单的请求规范调用API即可;
4. 稳定性和可靠性:middleware基于Golang语言实现,具有高效的并发处理能力和稳定的运行性能,保证了API服务的稳定性和可靠性;
5. 易于维护:middleware采用了Beego框架,具有清晰的代码结构和模块化设计,易于维护和扩展,保证了项目的长期可持续发展。
其向GUI提供的主要API如 表2-2 所示:
\begin{bupttable}{middleware 主要API}{table_gui_modules}
    \begin{tabular}{|l|l|l|}
        \hline \textbf{API} & \textbf{路由} & \textbf{描述} \\
        \hline PostSDTxt2Img & /v1/pics/txt2img & 通过Stable Diffusion模型生成图片  \\
        \hline PostSDImg2Img & /v1/pics/img2img & 通过Stable Diffusion模型修改图片 \\
        \hline PostDALLE2Edit & /v1/pics/openai/img2img & 通过DALL-E2模型修改图片  \\
		\hline GetLoras & /v1/pics/loras & 获取可用的LoRa模型列表  \\
		\hline PostHuggingFaceImgSegment & /v1/pics/huggingface/segment & 获取图像分割结果  \\
		\hline PostGPT3Dot5Turbo & /v1/chat/gpt3dot5turbo & 调用GPT3.5Turbo  \\
		\hline PostGPT4 & /v1/chat/gpt4 & 调用GPT4  \\
		\hline PostChatGLM2\_6B & /v1/chat/glm2\_6b & 调用ChatGLM2-6B  \\
		\hline PostGPT4V & /v1/chat/gpt4v & 调用GPT4V  \\
        \hline
    \end{tabular}
\end{bupttable}

\section{Stable Diffusion}
由于本项目对于图像生成模型的要求较高且需求复杂,为了便于结合Stable Diffusion模型和其他前沿研究成果和开源社区项目,本项目在构建Stable Diffusion模块时
以开源项目stable-diffusion-webui\footnote{https://github.com/AUTOMATIC1111/stable-diffusion-webui}为基础,
结合Control Net\cite{zhang2023adding}和基于DeepFake\cite{van2021deepfake}的开源项目sd-webui-roop\footnote{https://github.com/s0md3v/sd-webui-roop},
通过 API 为middleware提供服务。

\section{OpenAI}
本项目使用了OpenAI\footnote{https://openai.com}的GPT3.5Turo、GPT4、GPT4V、DALL-E2等模型,通过 API 调用OpenAI的模型。
% TODO: unimpl

\section{ChatGLM2-6B}
本项目使用开源的ChatGLM2-6B模型,使用开源项目LLaMA-Factory\footnote{https://github.com/hiyouga/LLaMA-Factory},
利用本项目提供的数据自动生成功能所生成的数据集,使用LoRa\cite{hu2021lora}方法对模型进行微调以在本项目所需的任务中获得更佳的表现。微调后的模型通过fastapi提供 API 服务。

\chapter{实现方法}
\section{图像自动遮罩与优化}

该项目提供了两种自动生成遮罩的方法:基于关键词对自动生成遮罩和基于已给出的点自动填充生成遮罩。
两种方法都会首先使用图像分割模型对图像进行分割(如图3-1),然后生成原始的遮罩,最后会通过本项目设计的优化算法生成最终的遮罩。
\begin{figure}[!htbp]
    \centering
    \subfloat[]{ %[]对齐方式,t为top,b为bottom,留空即可
	\label{Fig:SegmentOrigin1} % 子图1标签名
    	\includegraphics[width=0.45\textwidth]{pictures/example1_origin} %插入图片命令,格式为[配置]{图片路径}
    }
    \quad %空格
    \subfloat[]{
	\label{Fig:SegmentResult1} % 子图2标签名
    	\includegraphics[width=0.45\textwidth]{pictures/example1_segment}
    }
    \caption{图像分割结果:\protect\subref{Fig:SegmentOrigin1}原始图像,\protect\subref{Fig:SegmentResult1}分割结果} %注意须使用\protect\subref{}进行标号引用
    \label{Fig:Segment} % 整个组图的标签名
\end{figure}

\newpage
\subsection{图像自动遮罩}
基于关键词自动生成遮罩的方法会根据关键词和图像分割结果生成自动原始的遮罩,该功能会遍历每个给出的关键词,
若关键词与分割结果之一吻合,则会对相应的分割区域进行遮罩,生成原始的遮罩(如图3-2)。
\begin{figure}[!htbp]
    \centering
    \subfloat[]{ %[]对齐方式,t为top,b为bottom,留空即可
	\label{Fig:SegmentOrigin2} % 子图1标签名
    	\includegraphics[width=0.24\textwidth]{pictures/example2_origin} %插入图片命令,格式为[配置]{图片路径}
    }
    \quad %空格
    \subfloat[]{
	\label{Fig:SegmentResult2} % 子图2标签名
    	\includegraphics[width=0.24\textwidth]{pictures/example2_step1}
    }
    \caption{关键词自动生成遮罩结果:\protect\subref{Fig:SegmentOrigin2}原始图像,\protect\subref{Fig:SegmentResult2}kewords=['Background', 'Upper-clothes', 'Dress', 'Right-arm']得到的遮罩} %注意须使用\protect\subref{}进行标号引用
    \label{Fig:SegmentKeywords} % 整个组图的标签名
\end{figure}

基于已给出的点自动填充生成遮罩的方法会根据在图片中标记的点和图像分割结果生成自动原始的遮罩,该功能会遍历每个给出的点,
将该点所在的部分全部进行遮罩,最后生成原始的遮罩(如图3-3)。
\begin{figure}[!htbp]
    \centering
    \subfloat[]{
	\label{Fig:SegmentPoints} % 子图2标签名
    	\includegraphics[width=0.24\textwidth]{pictures/example1_points}
    }
    \quad %空格
    \subfloat[]{
	\label{Fig:SegmentStep1} % 子图2标签名
    	\includegraphics[width=0.24\textwidth]{pictures/example1_step1}
    }
    \caption{基于已给出的点自动填充生成遮罩:\protect\subref{Fig:SegmentPoints}标记后的图像,\protect\subref{Fig:SegmentStep1}生成的遮罩} %注意须使用\protect\subref{}进行标号引用
    \label{Fig:Point} % 整个组图的标签名
\end{figure}

\newpage
\subsection{对自动生成的遮罩进行优化}
由于分割模型性能的限制,生成的原始遮罩可能在某些细节上表现不佳而影响图像编辑模型的结果,因此设计了一个算法对自动生成的遮罩
进行优化。该算法可以根据配置文件的设置,对特定的未被遮罩的部分在遮罩的边缘进行收缩。

\begin{algorithm} 
	\begin{spacing}{1.3}
		\floatname{algorithm}{算法}
		\caption{遮罩优化算法} 
		\label{MaskOptimizationAlgorithm}
		\renewcommand{\algorithmicrequire}{\textbf{输入:}}
		\renewcommand{\algorithmicensure}{\textbf{输出:}} 
			\begin{algorithmic}[1] 
				\Require 原始遮罩$OriginMask$,图像分割结果$SegmentResult$,配置文件$Config$
				\Ensure 优化后的遮罩$OptimizedMask$
				\State 获取遮罩与非遮罩的描边得到像素$EdgePixcels$
				\State 从配置文件和图像分割结果获取$ConfigPixcels$
                \State 仅保留出现在$EdgePixcels$中的$ConfigPixcels$
                \For{$pixcel$ in $ConfigPixcels$}
                    \State Apply MinFilter $Kerne$(in $Config$) in $OriginMask$[$pixcel$]
                \EndFor
                \State 得到优化后的遮罩$OptimizedMask$
			\end{algorithmic}
	\end{spacing}
\end{algorithm}

该算法实现的效果如 图3-4 所示,可见在发丝附近遮罩的质量得到了明显的改善。
\begin{figure}[!htbp]
    \centering
    \subfloat[]{ %[]对齐方式,t为top,b为bottom,留空即可
	\label{Fig:MaskOrigin} % 子图1标签名
    	\includegraphics[width=0.24\textwidth]{pictures/example2_step1} %插入图片命令,格式为[配置]{图片路径}
    }
    \quad %空格
    \subfloat[]{
	\label{Fig:MaskOptimized} % 子图2标签名
    	\includegraphics[width=0.24\textwidth]{pictures/example2_step2}
    }
    \caption{遮罩优化结果:\protect\subref{Fig:MaskOrigin}原始遮罩,\protect\subref{Fig:MaskOptimized}算法优化后的遮罩} %注意须使用\protect\subref{}进行标号引用
    \label{Fig:MaskOptimize} % 整个组图的标签名
\end{figure}
\chapter{基础模块示例}

\section{特殊文本类型}
\subsection{脚注}
% 如果你的项目来源于科研项目,可以使用以下指令插入无编号脚注于正文第一页
\blfootnote{本项目来源于科研项目“基于\LaTeX{}的本科毕业设计”,项目编号1124}
社交媒体是一种供用户创建在线社群来分享信息、观点、个人信息和其它内容(如视频)的电子化交流平台,社交网络服务(social network service, SNS)和微博客(microblogging)都属于社交媒体的范畴\cite{webster_social_media},国外较为知名的有Facebook\footnote{http://www.facebook.com/}、Instagram\footnote{https://www.instagram.com/}、Twitter\footnote{http://www.twitter.com/}、LinkedIn\footnote{http://www.linkedin.com/}等,国内较为知名的有新浪微博\footnote{http://www.weibo.com/}。

在社交媒体的强覆盖下,新闻信息的传播渠道也悄然了发生变化。\cite{false_news_spread_2018}

\subsection{定义、定理与引理等}
\begin{definition}
这是一条我也不知道在说什么的定义,反正我就是写在这里做个样子罢了,也没人会仔细读。\cite{周兴2017基于深度学习的谣言检测及模式挖掘}
\end{definition}

\begin{theorem}
这是一条我也不知道在说什么的定理,反正我就是写在这里做个样子罢了,也没人会仔细读。
\end{theorem}

\begin{axiom}
这是一条我也不知道在说什么的公理,反正我就是写在这里做个样子罢了,也没人会仔细读。
\end{axiom}

\begin{lemma}
这是一条我也不知道在说什么的引理,反正我就是写在这里做个样子罢了,也没人会仔细读。
\end{lemma}

\begin{proposition}
这是一条我也不知道在说什么的命题,反正我就是写在这里做个样子罢了,也没人会仔细读。
\end{proposition}

\begin{corollary}
这是一条我也不知道在说什么的推论,反正我就是写在这里做个样子罢了,也没人会仔细读。
\end{corollary}

\subsection{中英文文献、学位论文引用}
根据美国皮尤研究中心的2017年9月发布的调查结果\cite{pew_news_use_2017},67\%的美国民众会从社交媒体上获取新闻信息,其中高使用频率用户占20\%。在国内,中国互联网信息中心《2016年中国互联网新闻市场研究报告》\cite{internet_news_2016}也显示,社交媒体已逐渐成为新闻获取、评论、转发、跳转的重要渠道,在2016年下半年,曾经通过社交媒体获取过新闻资讯的用户比例高达90.7\%,在微信、微博等社交媒体参与新闻评论的比例分别为62.8\%和50.2\%。社交媒体正在成为网络上热门事件生成并发酵的源头,在形成传播影响力后带动传统媒体跟进报道,最终形成更大规模的舆论浪潮。\cite{Yang2012Automatic}

在国内,新浪微博由于其发布方便、传播迅速、受众广泛且总量大的特点,成为了虚假信息传播的重灾区:《中国新媒体发展报告(2013)》\cite{唐绪军2013中国新媒体发展报告}显示,2012年的100件微博热点舆情案例中,有超过1/3出现谣言;《中国新媒体发展报告(2015)》\cite{唐绪军2015中国新媒体发展报告}对2014年传播较广、比较典型的92条假新闻进行了多维度分析,发现有59\%的虚假新闻首发于新浪微博。

此等信息的传播严重损害了有关公众人物的名誉权,降低了社交媒体服务商的商业美誉度,扰乱了网络空间秩序,冲击着网民的认知,极易对民众造成误导,带来诸多麻烦和经济损失,甚至会导致社会秩序的混乱。针对社交媒体谣言采取行动成为了有关部门、服务提供商和广大民众的共同选择。\cite{周兴2017基于深度学习的谣言检测及模式挖掘}

\section{图表及其引用}
此处引用了简单的表\ref{crowdwisdom_TMP}。

请注意,\LaTeX{}的图表排版规则决定了图表\textbf{不一定会乖乖呆在你插入的地方},这是为了避免Word中由于图片尺寸不匹配在页面下部出现的的空白,所以请不要使用“下图”“下表”作为指向文字,应使用“图1-1所示”这样的表述。

\begin{bupttable}{基于浏览者行为的特征}{crowdwisdom_TMP}

    \begin{tabular}{l|l|l}
		\hline \textbf{特征} & \textbf{描述} & \textbf{形式与理论范围}\\
		\hline 点赞量 & 微博的点赞数量 & 数值,$\mathbb{N}$ \\
		\hline 评论量 & 微博的评论数量 & 数值,$\mathbb{N}$ \\
		\hline 转发量 & 微博的转发数量 & 数值,$\mathbb{N}$ \\
		\hline
    \end{tabular}
\end{bupttable}

此处引用了复杂的表\ref{complexcrowdwisdom_TMP}。


\begin{bupttable}{基于浏览者行为的复杂特征}{complexcrowdwisdom_TMP}
    \begin{tabular}{l|l|l|l}
        \hline
        \multicolumn{1}{c|}{\multirow{2}{*}{\textbf{类别}}} & \multicolumn{1}{c|}{\multirow{2}{*}{\textbf{特征}}} & \multicolumn{2}{c}{\textbf{不知道叫什么的表头}} \\
        \cline{3-4}
        & & \multicolumn{1}{c|}{\textbf{描述}} & \multicolumn{1}{c}{\textbf{形式与理论范围}} \\
        \hline
        \multirow{3}{*}{正常互动} & 点赞量 & 微博的点赞数量 & 数值,$\mathbb{N}$ \\
        \cline{2-4}
        & 评论量 & 微博的评论数量 & 数值,$\mathbb{N}$ \\
        \cline{2-4}
        & 转发量 & 微博的转发数量 & 数值,$\mathbb{N}$ \\
        \hline
        非正常互动 & 羡慕量 & 微博的羡慕数量 & 数值,$\mathbb{N}$ \\
        \hline
    \end{tabular}
\end{bupttable}

此处展示了更专业的表\ref{tab:abbr_table},一个好的表格没有竖线。
% 请注意1)tabularx环境对多行文本的处理;2)booktabs宏包中支持的更粗的顶端和底端表格边界线,边界线与文本间更大的间距。
\begin{bupttable}{红警2名词解释}{tab:abbr_table}
    \begin{tabularx}{\textwidth}{llX}
        \toprule
        \textbf{术语类别} & \textbf{缩略语} & \textbf{解释} \\ \midrule
        & 兵营 & 兵营(Barracks),《命令与征服\ 红色警戒2:尤里的复仇》游戏中的一种生产建筑,用以生产步兵单位 \\ \cmidrule(l){2-3}
        & 建造场 & 建造场(Construction Yard),《命令与征服\ 红色警戒2:尤里的复仇》游戏中的一种基础建筑,用以支持其他建筑的建造 \\ \cmidrule(l){2-3}
        & 矿厂 & 矿石精炼厂(Ore Refinery),《命令与征服\ 红色警戒2:尤里的复仇》游戏中的一种资源建筑,用以将矿车采集的矿石转化为游戏资金 \\ \cmidrule(l){2-3}
        游戏 & 空指 & 空指部(Airforce Command Headquarters),《命令与征服\ 红色警戒2:尤里的复仇》游戏中的一种资源建筑,用以提供雷达功能和T2科技及生产部分空军单位 \\ \cmidrule(l){2-3}
        & 相机 & 游戏术语,特指游戏内的观察区域和视角 \\ \cmidrule(l){2-3}
        & 重工 & 战车工厂(War Factory),《命令与征服\ 红色警戒2:尤里的复仇》游戏中的一种生产建筑,用以生产载具单位 \\ \cmidrule(l){2-3}
        & 战争迷雾 & 游戏术语,《命令与征服\ 红色警戒2:尤里的复仇》中指黑色的未探索区域 \\ \bottomrule
    \end{tabularx}
\end{bupttable}

此处引用了一张图。图\ref{autoencoder_TMP}表示的是一个由含有4个神经元的输入层、含有3个神经元的隐藏层和含有4个神经元的输出层组成的自编码器,$+1$代表偏置项。

%图片宽度设置为文本宽度的75%,可以调整为合适的比例
\buptfigure[width=0.7\textwidth]{pictures/autoencoder}{自编码器结构}{autoencoder_TMP}

%组图示例,已按照指导手册要求设计,由于子图数量不同,无法压缩成\buptfigure那样,大家对照示例即可
\begin{figure}[!htbp]
    \centering
    \subfloat[]{ %[]对齐方式,t为top,b为bottom,留空即可
	\label{Fig:R1} % 子图1标签名
    	\includegraphics[width=0.45\textwidth]{pictures/autoencoder} %插入图片命令,格式为[配置]{图片路径}
    }
    \quad %空格
    \subfloat[]{
	\label{Fig:R2} % 子图2标签名
    	\includegraphics[width=0.45\textwidth]{pictures/autoencoder}
    }
    \caption{这是两个自编码器结构,我就是排一下子图的效果:\protect\subref{Fig:R1}左边的自编码器,\protect\subref{Fig:R2}右边的自编码器} %注意须使用\protect\subref{}进行标号引用
    \label{Fig:RecAccuracy} % 整个组图的标签名
\end{figure}

\section{公式与算法表示}

\subsection{例子:基于主成分分析}

\subsubsection{主成分分析算法}

下面对主成分分析进行介绍。

主成分分析是一种简单的机器学习算法,其功能可以从两方面解释:一方面可以认为它提供了一种压缩数据的方式,另一方面也可以认为它是一种学习数据表示的无监督学习算法。\cite{Goodfellow2016DeepLearning}
通过PCA,我们可以得到一个恰当的超平面及一个投影矩阵,通过投影矩阵,样本点将被投影在这一超平面上,且满足最大可分性(投影后样本点的方差最大化),直观上讲,也就是能尽可能分开。

对中心化后的样本点集$\bm{X}=\{\bm{x}_1,\bm{x}_2,\ldots,\bm{x}_i,\ldots,\bm{x}_m\}$(有$\sum_{i=1}^{m}\bm{x}_i = 0$),考虑将其最大可分地投影到新坐标系\ $\bm{W}= \{\bm{w}_1,\bm{w}_2,\ldots,\bm{w}_i,\ldots,\bm{w}_d\} $,其中$\bm{w}_i$是标准正交基向量,满足$\|\bm{w}_i\|_2 = 1$, $\bm{w}_i^T\bm{w}_j = 0$($i \not= j$)。假设我们需要$d^\prime$($d^\prime < d$)个主成分,那么样本点$\bm{x}_i$在低维坐标系中的投影是$\bm{z}_i = (z_{i1};z_{i2};\ldots;z_{id^\prime})$,其中$z_{ij} = \bm{w}_j^\mathrm{T}\bm{x}_i$,是$\bm{x}_i$在低维坐标系下第$j$维的坐标。
对整个样本集,投影后样本点的方差是
\begin{equation}
\begin{aligned}
    & \frac{1}{m}\sum_{i=1}^m \bm{z}_i^\mathrm{T}\bm{z}_i \\
= & \frac{1}{m}\sum_{i=1}^m (\bm{x}_i^\mathrm{T}\bm{W})^\mathrm{T}(\bm{x}_i^\mathrm{T}\bm{W}) \\
= & \frac{1}{m}\sum_{i=1}^m \bm{W}^\mathrm{T}\bm{x}_i\bm{x}_i^\mathrm{T}\bm{W} \\
= & \frac{1}{m} \bm{W}^\mathrm{T}\bm{X}\bm{X}^\mathrm{T}\bm{W} \\
\end{aligned}
\end{equation}

由于我们知道新坐标系$\bm{W}$的列向量是标准正交基向量,且样本点集$\bm{X}$已经过中心化,则PCA的优化目标可以写为
\begin{equation}
\label{PCA_goal_TMP}
\begin{aligned}
& \max_{\substack{\bm{W}}}  &  tr(\bm{W}^\mathrm{T}\bm{X}\bm{X}^ \mathrm{T}\bm{W}) \\
& \operatorname{ s.t. }  &  \bm{W}^\mathrm{T}\bm{W} = \bm{I} \\
\end{aligned}
\end{equation}

由于$\bm{X}\bm{X}^ \mathrm{ T }$是协方差矩阵,那么只需对它做特征值分解,即
\begin{equation}
\label{PCA_eigenvalue}
\bm{X}^ \mathrm{ T }\bm{X} = \bm{W}\bm{\Lambda}\bm{W}^ \mathrm{ T } \\
\end{equation}
其中$\bm{\Lambda}=diag(\bm{\lambda})$,$\bm{\lambda} = \{\lambda_1,\lambda_2,\ldots,\lambda_m\}$。

具体地,考虑到它是半正定矩阵的二次型,存在最大值,可对\eqref{PCA_goal_TMP}使用拉格朗日乘数法
\begin{equation}
\bm{X}\bm{X}^ \mathrm{ T }\bm{w}_i  = \lambda_i \bm{w}_i \\
\end{equation}

之后将求得的特征值降序排列,取前$d^\prime$个特征值对应的特征向量组成所需的投影矩阵$\bm{W}^\prime =(\bm{w}_1,\bm{w}_2,\ldots,\bm{w}_{d^\prime})$,即可得到PCA的解。PCA算法的描述如算法\ref{PCA_algorithm}所示。

\begin{algorithm} 
	\begin{spacing}{1.3}
		\floatname{algorithm}{算法}
		\caption{主成分分析(PCA)} 
		\label{PCA_algorithm}
		\renewcommand{\algorithmicrequire}{\textbf{输入:}}
		\renewcommand{\algorithmicensure}{\textbf{输出:}} 
		\begin{algorithmic}[1] 
			\Require 样本集$\bm{x}=\{\bm{x}_1,\bm{x}_2,\ldots,\bm{x}_i,\ldots,\bm{x}_m\}$,低维空间维数$d^\prime$ 
			\Ensure 投影矩阵  $\bm{W}^\prime =(\bm{w}_1,\bm{w}_2,\ldots,\bm{w}_{d^\prime})$
			\State 对所有样本中心化$\bm{x}_i \gets \bm{x}_i - \frac{1}{m}\sum_{i=1}^m \bm{x}_i$
			\State  计算样本的协方差$\bm{X}\bm{X}^ \mathrm{T}$
			\State 对协方差矩阵$\bm{X}\bm{X}^ \mathrm{T}$做特征值分解
			\State 取最大的$d^\prime$个特征值所对应的特征向量$\bm{w}_1,\bm{w}_2,\ldots,\bm{w}_{d^\prime}$
		\end{algorithmic}  
	\end{spacing}
\end{algorithm}

\subsubsection{主成分分析可信度评估方法}
记待判定微博$\bm{w}_0$的经典特征向量为$\bm{f}^{c}_{0}$,它的发布者在$\bm{w_0}$前发布的$k$条微博为$\bm{W} = \bm{w}_1,\bm{w}_2,\ldots,\bm{w}_k$,这$k$条微博对应的经典特征向量集为$\bm{F}^{c}_{W} = \{ \bm{f}^{c}_{1},\bm{f}^{c}_{2},\ldots,\bm{f}^{c}_{k} \}$。令$label = 1$代表谣言,$label = 0$代表非谣言。算法的具体流程如算法\ref{PCA_model}所示。

\begin{algorithm}
	\begin{spacing}{1.3}
		\floatname{algorithm}{算法}
		\caption{基于PCA的信息可信度评估} 
		\label{PCA_model}
		\renewcommand{\algorithmicrequire}{\textbf{输入:}}
		\renewcommand{\algorithmicensure}{\textbf{输出:}} 
			\begin{algorithmic}[1] 
				\Require $\bm{f}^{c}_{0}$,$\bm{F}^{c}_{W}$,保留主成分数$n$
				\Ensure 标签$label\in \{0,1\}$
				\State 对所有特征向量应用PCA,保留前$n$个主成分$\bm{o}^{c}_{i} \gets PCA(\bm{f}^{c}_{i}, n)$($i = 0,1,\ldots,k$)
				\State 计算$\bm{F}^{c}_{W}$中各向量的平均距离$\mu$和标准差$\sigma$
				\State 计算阈值$thr = {\mu} / {\sigma}$
				\If {$\min_{1<j\le k} \|\bm{o}^{c}_{0} - \bm{o}^{c}_{j} \|_2 > thr$}
					\State $ label \gets 1 $
				\Else
					\State $ label \gets 0 $
				\EndIf
			\end{algorithmic}
	\end{spacing}
\end{algorithm}

\section{代码表示}

%据悉以下语言被lstlisting支持:Awk, bash, Basi4, C#, C++, C, Delphi, erlang, Fortran, GCL, Haskell, HTML, Java, JVMIS, Lisp, Logo, Lua, make, Mathematica, Matlab, Objective C , Octave, Pascal, Perl, PHP, Prolog,  Python, R, Ruby, SAS, Scilab, sh, SHELXL, Simula, SQL, tcl, TeX, VBScript, Verilog, VHDL, XML, XSLT
%遗憾的是,JavaScript不被支持,请上网搜索支持该语言的方法

\subsection{直接书写代码在.tex中}
下面的代码\ref{plus}是用Python编写的加法函数。

\begin{lstlisting}[language=Python, caption=加法, label=plus, tabsize=2]  
def plusFunc(a, b):
	return a + b 
\end{lstlisting}  

\subsection{引用代码文件}
下面的代码\ref{recursion}是用Python文件中引入的倒序打印$x$到$1$的函数,请查看code文件夹。

\lstinputlisting[language=Python, caption=倒序打印数字, label=recursion, tabsize=2]{code/recursion.py}

\section{列表样式}

\subsection{使用圆点作为项目符号}

\begin{itemize}
\item \textbf{第一章为基础模块示例},是的,本章的名字就是基础模块示例,正如你看到这个样子。
\item \textbf{第二章为不存在},是的,其实它不存在。
\end{itemize}

\subsection{使用数字作为项目符号}

\begin{enumerate}
\item \textbf{第一章为基础模块示例},是的,本章的名字就是基础模块示例,正如你看到这个样子。
\item \textbf{第二章为不存在},是的,其实它不存在。
\end{enumerate}

\subsection{句中数字编号列表样式}

\begin{enumerate*}
    \item \textbf{第一章为基础模块示例},是的,本章的名字就是基础模块示例,正如你看到这个样子;
    \item \textbf{第二章为不存在},是的,其实它不存在。
\end{enumerate*}

\chapter{为了目录撑到第二页}
\section{我不得不再添加一点内容}
\section{尽管这些章节一点正文都没有}
\section{是的}
\section{真的没有}
\section{我已经不知道说什么了}
\section{如果有,我们就祝愿一下学校教务处什么时候转变一下思维}
\section{把控制格式这种事情往前做}
\section{不要总是觉得折磨学生是合理的}
\section{你拿着教学管理岗位的工资}
\section{你需要折磨一下你自己才对}
\section{不要觉得我对别人要求太高,对自己太低}
\section{我对自己要求低的话也不至于想要修订这份模板}
%%%%%%%%%%%%%%%%%%%%%%% Main Area ENDs Here %%%%%%%%%%%%%%%%%%%%%%%%
%\let\cleardoublepage=\cleardoublepagebak

\begin{nopagenumber}
% Reference
\clearpage\phantomsection\addcontentsline{toc}{chapter}{参考文献}
\bibliographystyle{buptbachelor}
\refbodyfont{\bibliography{ref}}

% Thanks to page
\clearpage
\chapter{致\qquad{}谢}
\normalsize\thankwords

% Appendix
\setcounter{figure}{0} 
\renewcommand{\thefigure}{~附-\arabic{figure}~}
\setcounter{equation}{0} 
\renewcommand{\theequation}{~附-\arabic{equation}~}
\setcounter{table}{0} 
\renewcommand{\thetable}{~附-\arabic{table}~}
\setcounter{lstlisting}{0} 
\makeatletter
  \renewcommand \thelstlisting
       {附-\@arabic\c@lstlisting}
\makeatother


\chapter*{附\qquad{}录}
\phantomsection\addcontentsline{toc}{chapter}{附\qquad{}录}

\phantomsection\addcontentsline{toc}{section}{附录1\quad{}缩略语表}
\section*{附录1\quad{}缩略语表}

\begin{bupttable}{基于浏览者行为的特征}{crowdwisdom2}
    \begin{tabular}{l|l|l}
        \hline \textbf{特征} & \textbf{描述} & \textbf{形式与理论范围}\\
        \hline 点赞量 & 微博的点赞数量 & 数值,$\mathbb{N}$ \\
        \hline 评论量 & 微博的评论数量 & 数值,$\mathbb{N}$ \\
        \hline 转发量 & 微博的转发数量 & 数值,$\mathbb{N}$ \\
        \hline
    \end{tabular}
\end{bupttable}

\begin{bupttable}{基于浏览者行为的复杂特征}{complexcrowdwisdom2}
    \begin{tabular}{l|l|l|l}
		\hline
        \multicolumn{1}{c|}{\multirow{2}{*}{\textbf{类别}}} & \multicolumn{1}{c|}{\multirow{2}{*}{\textbf{特征}}} & \multicolumn{2}{c}{\textbf{不知道叫什么的表头}} \\
        \cline{3-4}
         & & \multicolumn{1}{c|}{\textbf{描述}} & \multicolumn{1}{c}{\textbf{形式与理论范围}} \\
		\hline
        \multirow{3}{*}{正常互动} & 点赞量 & 微博的点赞数量 & 数值,$\mathbb{N}$ \\
		\cline{2-4}
         & 评论量 & 微博的评论数量 & 数值,$\mathbb{N}$ \\
		\cline{2-4}
         & 转发量 & 微博的转发数量 & 数值,$\mathbb{N}$ \\
		\hline
        非正常互动 & 羡慕量 & 微博的羡慕数量 & 数值,$\mathbb{N}$ \\
        \hline
    \end{tabular}
\end{bupttable}
\buptfigure[width=0.15\textheight]{pictures/autoencoder}{自编码器结构}{autoencoder}

\begin{lstlisting}[language=Python, caption=减法, label=minus, tabsize=2]  
def minusFunc(a, b):
	return a - b 
\end{lstlisting}  

\begin{equation}
\label{PCA_goal}
\begin{aligned}
\max_{\substack{\bm{W}}}  &  tr(\bm{W}^\mathrm{T}\bm{X}\bm{X}^ \mathrm{T}\bm{W})
\end{aligned}
\end{equation}

\clearpage
\phantomsection\addcontentsline{toc}{section}{附录2\quad{}数学符号}
\section*{附录2\quad{}数学符号}
\begin{center}
	\begin{tabular}{ccc}
		\multicolumn{2}{c}{\textbf{数和数组}} \\
		\\
		$a$ & 标量(整数或实数)\\
		$\bm{a}$ & 向量\\
		$dim()$ & 向量的维数\\
		$\bm{A}$ & 矩阵\\
		$\bm{A}^\mathrm{T}$ & 矩阵$\textbf{A}$的转置\\
		$\bm{I}$ & 单位矩阵(维度依据上下文而定) \\
 		$diag(\bm{a})$ & 对角方阵,其中对角元素由向量$\bm{a}$确定 \\

	\end{tabular}
\end{center}


% Translated Article
\newpage\backmatter
\thispagestyle{empty}
\phantomsection\addcontentsline{toc}{chapter}{外\quad{}文\quad{}资\quad{}料}
% 原文第一页,PDF缩放比例为0.95,可以自行调整
\includepdf[pages=1, scale=0.95, pagecommand=\begin{center}\translationtitlefont{外\quad{}文\quad{}资\quad{}料}\end{center}]{docs/translation.pdf}
% 原文剩余部分
\includepdf[pages=2-, scale=0.95]{docs/translation.pdf}

% Translation
\setcounter{chapter}{0}
\renewcommand{\thefigure}{~外\arabic{chapter}-\arabic{figure}~}
\renewcommand{\theequation}{~外\arabic{chapter}-\arabic{equation}~}
\renewcommand{\thetable}{~外\arabic{chapter}-\arabic{table}~}
\renewcommand{\thelstlisting}{~外\arabic{chapter}-\arabic{lstlisting}~}

\begin{center}
\phantomsection\addcontentsline{toc}{chapter}{外\quad{}文\quad{}译\quad{}文}
\translationtitlefont{外\quad{}文\quad{}译\quad{}文}
\end{center}
\vspace{8mm}
\thispagestyle{empty}


\begin{center}
\sanhao\heiti\textbf{真假新闻的在线传播}

\xiaosihao\songti{Soroush Vosoughi, Deb Roy, Sinan Aral}

\xiaosihao\songti{麻省理工学院}
\end{center}

\songti{}
\begingroup % 限制两个let语句的作用范围在外文译文部分
\let\clearpage\relax
\let\cleardoublepage\relax

%以下是排版示例,在这里为了使章节编号不出现在目录中,使用了无编号的样式,代价是这些数字都要自己书写。

\chapter*{第一章\quad{}概述}
%每一个chapter后记得以下两行
\newtranschapter

\section*{1.1\quad{}概述}
决策、合作、通信和市场领域的基础理论全都将对真实或准确度的概念化作为几乎一切人类努力的核心。然而,不论是真实信息还是虚假信息都会于在线媒体上迅速传播。定义什么是真、什么是假成了一种常见的政治策略,而不是基于一些各方同意的事实的争论。我们的经济也难免遭受虚假信息传播的影响。虚假流言会影响股价和大规模投资的动向,例如,在一条声称巴拉克·奥巴马在爆炸中受伤的推文发布后,股市市值蒸发了1300亿美元。的确,从自然灾害到恐怖袭击,我们对一切事情的反应都受到了扰乱。

新的社交网络技术在使信息的传播速度变快和规模变大的同时,也便利了不实信息(即不准确或有误导性的信息)的传播。然而,尽管我们对信息和新闻的获取越来越多地收到这些新技术的引导,但我们仍然对他们在虚假信息传播上的作用知之甚少。尽管媒体对假新闻传播的轶事分析给予了相当多的关注,但仍然几乎没有针对不实信息扩散或其发布源头的大规模实证调查。目前,虚假信息传播的研究仅仅局限于小的、局部的样本的分析上,而这些分析忽略了两个最重要的科学问题:真实信息和虚假信息的传播有什么不同?哪些人类判断中的因素可以解释这些不同?

\begin{equation}
\label{PCA_goal_appx1}
\begin{aligned}
\max_{\substack{\bm{W}}}  &  tr(\bm{W}^\mathrm{T}\bm{X}\bm{X}^ \mathrm{T}\bm{W})
\end{aligned}
\end{equation}

我只是为了把第二章挤到下一页而凑的字。我只是为了把第二章挤到下一页而凑的字。我只是为了把第二章挤到下一页而凑的字。我只是为了把第二章挤到下一页而凑的字。我只是为了把第二章挤到下一页而凑的字。我只是为了把第二章挤到下一页而凑的字。我只是为了把第二章挤到下一页而凑的字。我只是为了把第二章挤到下一页而凑的字。我只是为了把第二章挤到下一页而凑的字。我只是为了把第二章挤到下一页而凑的字。我只是为了把第二章挤到下一页而凑的字。我只是为了把第二章挤到下一页而凑的字。我只是为了把第二章挤到下一页而凑的字s。我只是为了把第二章挤到下一页而凑的字。我只是为了把第二章挤到下一页而凑的字。我只是为了把第二章挤到下一页而凑的字。我只是为了把第二章挤到下一页而凑的字。我只是为了把第二章挤到下一页而凑的字。我只是为了把第二章挤到下一页而凑的字。我只是为了把第二章挤到下一页而凑的字。我只是为了把第二章挤到下一页而凑的字。我只是为了把第二章挤到下一页而凑的字。我只是为了把第二章挤到下一页而凑的字。我只是为了把第二章挤到下一页而凑的字。

\newpage %每一章需要另起一页,为了灵活,我没有把它固定在样式中,你可以根据需求添加分页符
\chapter*{第二章\quad{}我也不知道是什么}
\newtranschapter

新的社交网络技术在使信息的传播速度变快和规模变大的同时,也便利了不实信息(即不准确或有误导性的信息)的传播。然而,尽管我们对信息和新闻的获取越来越多地收到这些新技术的引导,但我们仍然对他们在虚假信息传播上的作用知之甚少。尽管媒体对假新闻传播的轶事分析给予了相当多的关注,但仍然几乎没有针对不实信息扩散或其发布源头的大规模实证调查。目前,虚假信息传播的研究仅仅局限于小的、局部的样本的分析上,而这些分析忽略了两个最重要的科学问题:真实信息和虚假信息的传播有什么不同?哪些人类判断中的因素可以解释这些不同?

新的社交网络技术在使信息的传播速度变快和规模变大的同时,也便利了不实信息(即不准确或有误导性的信息)的传播。然而,尽管我们对信息和新闻的获取越来越多地收到这些新技术的引导,但我们仍然对他们在虚假信息传播上的作用知之甚少。尽管媒体对假新闻传播的轶事分析给予了相当多的关注,但仍然几乎没有针对不实信息扩散或其发布源头的大规模实证调查。目前,虚假信息传播的研究仅仅局限于小的、局部的样本的分析上,而这些分析忽略了两个最重要的科学问题:真实信息和虚假信息的传播有什么不同?哪些人类判断中的因素可以解释这些不同?

新的社交网络技术在使信息的传播速度变快和规模变大的同时,也便利了不实信息(即不准确或有误导性的信息)的传播。然而,尽管我们对信息和新闻的获取越来越多地收到这些新技术的引导,但我们仍然对他们在虚假信息传播上的作用知之甚少。尽管媒体对假新闻传播的轶事分析给予了相当多的关注,但仍然几乎没有针对不实信息扩散或其发布源头的大规模实证调查。目前,虚假信息传播的研究仅仅局限于小的、局部的样本的分析上,而这些分析忽略了两个最重要的科学问题:真实信息和虚假信息的传播有什么不同?哪些人类判断中的因素可以解释这些不同?

\begin{equation}
\label{PCA_goal_appx2}
\begin{aligned}
\max_{\substack{\bm{W}}}  &  tr(\bm{W}^\mathrm{T}\bm{X}\bm{X}^ \mathrm{T}\bm{W})
\end{aligned}
\end{equation}

新的社交网络技术在使信息的传播速度变快和规模变大的同时,也便利了不实信息(即不准确或有误导性的信息)的传播。然而,尽管我们对信息和新闻的获取越来越多地收到这些新技术的引导,但我们仍然对他们在虚假信息传播上的作用知之甚少。尽管媒体对假新闻传播的轶事分析给予了相当多的关注,但仍然几乎没有针对不实信息扩散或其发布源头的大规模实证调查。目前,虚假信息传播的研究仅仅局限于小的、局部的样本的分析上,而这些分析忽略了两个最重要的科学问题:真实信息和虚假信息的传播有什么不同?哪些人类判断中的因素可以解释这些不同?

新的社交网络技术在使信息的传播速度变快和规模变大的同时,也便利了不实信息(即不准确或有误导性的信息)的传播。然而,尽管我们对信息和新闻的获取越来越多地收到这些新技术的引导,但我们仍然对他们在虚假信息传播上的作用知之甚少。尽管媒体对假新闻传播的轶事分析给予了相当多的关注,但仍然几乎没有针对不实信息扩散或其发布源头的大规模实证调查。目前,虚假信息传播的研究仅仅局限于小的、局部的样本的分析上,而这些分析忽略了两个最重要的科学问题:真实信息和虚假信息的传播有什么不同?哪些人类判断中的因素可以解释这些不同?

新的社交网络技术在使信息的传播速度变快和规模变大的同时,也便利了不实信息(即不准确或有误导性的信息)的传播。然而,尽管我们对信息和新闻的获取越来越多地收到这些新技术的引导,但我们仍然对他们在虚假信息传播上的作用知之甚少。尽管媒体对假新闻传播的轶事分析给予了相当多的关注,但仍然几乎没有针对不实信息扩散或其发布源头的大规模实证调查。目前,虚假信息传播的研究仅仅局限于小的、局部的样本的分析上,而这些分析忽略了两个最重要的科学问题:真实信息和虚假信息的传播有什么不同?哪些人类判断中的因素可以解释这些不同?

\endgroup

% 开题报告
\blankmatter
\phantomsection\addcontentsline{toc}{chapter}{开\quad{}题\quad{}报\quad{}告}
\includepdf[pages=-]{docs/openingReport.pdf}

% 中期检查表
\blankmatter
\phantomsection\addcontentsline{toc}{chapter}{中\quad{}期\quad{}检\quad{}查\quad{}表}
\includepdf[pages=-]{docs/interimReport.pdf}

% 教师指导毕业设计(论文)记录表
\blankmatter
\phantomsection\addcontentsline{toc}{chapter}{教师指导毕业设计(论文)记录表}
\includepdf[pages=-]{docs/guidance.pdf}

\end{nopagenumber}

\end{document}
